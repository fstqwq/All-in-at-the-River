% !TEX root = ../main.tex
\begin{small}
\begin{itemize}[leftmargin=1mm]
    \setlength{\itemsep}{0pt}
    \setlength{\parskip}{0.5pt}
    \item 检查身份证件: 护照、学生证、胸牌以及现场所需通行证。
    \item 确认什么东西能带进场。特别注意: 智能手表、金属(钥匙)等等。
    \item 测试鼠标、键盘、显示器和座椅。如果有问题, 立刻联系工作人员。
    \item 确认比赛前能动什么, 不能动什么, 能存储什么配置文件。
    \item 测试比赛提交方式。如果有 \texttt{submit} 命令, 确认如何使用。讨论是否应该不用以避免 \texttt{submit > a.cpp}。
    \item 如果可以的话, 设置定期备份文件。
    \item 测试 OJ 栈大小。如果不合常理, 发 clar 问一下。
    \item 测试编译器版本。C++20 不能用 \texttt{cin >> (s + 1)}; C++17 \texttt{auto \lbrack x, y\rbrack : a}; C++14 \texttt{\lbrack \rbrack (auto x, auto y)}; C++11 \texttt{auto}; \texttt{bits/stdc++.h}; \texttt{pb\_ds}。
\begin{minted}{C++}
#include <ext/rope>
using namespace __gnu_cxx;
rope <int> R; R.insert(y, x); R[x]; R.erase(x, 1);
#include <ext/pb_ds/assoc_container.hpp> 
#include <ext/pb_ds/tree_policy.hpp> 
using namespace __gnu_pbds;
tree <int, null_type, less<int>, rb_tree_tag,
tree_order_statistics_node_update> s;
s.insert(1); s.find_by_order(0); s.order_of_key(5);
\end{minted}
    \item 测试 \texttt{\_\_int128, \_\_float128, long double}
    \item 测试代码长度限制; 测试 output limit; 测试 stderr limit。
    \item 测试内存限制: MLE 还是报 RE? 栈溢出呢? 
    \item 测试浮点数性能: FFT 能跑多快? 测试内存性能: 线段树、树状数组、素数筛能跑多快? 测试 CPU 性能: 阶乘、快速幂能跑多快? 记得开 O2。
    \item 测试 clar: 如果问不同类型的愚蠢的问题, 得到的回复是否不一样? 
    \item 测试 clock() 是否能够正常工作; 测试本地性能与提交性能。
    \item 测试本地是否有 fsan, gdb。\begin{minted}{text}
address, undefined, return, shift, integer-divide-by-zero,
bounds-strict, float-cast-overflow, builtin
\end{minted}
    \item 测试 Python, Java 本地环境与提交环境。Python 快吗? $A\times B$ 能跑多快? 输入输出呢? 
    \item 测试 \texttt{time} 命令是否能显示内存占用。
\begin{minted}{bash}
/usr/bin/time -v ./a.out
\end{minted}
\end{itemize}
提交前检查: 
\begin{itemize}[leftmargin=1mm]
    \setlength{\itemsep}{0pt}
    \setlength{\parskip}{0.5pt}
\item 保存, 编译, 测过样例
\item 数据范围: N/M, 输入小数, 模数, LL; 时间空间限制
\item 调用初始化函数, 清空时数组大小 (0, 1, n/m?, n+1)
\item 输入输出格式 (\texttt{\%Lf, \%llu}), 取模取到正值
\item 多组没读入完就 break
\item 关闭流同步, 注意不要关流同步混用
\item 需要 \texttt{assert} 的话第一次就加上
\item 开 \texttt{optimize("Ofast") target("avx2")}
\end{itemize}
提交后检查:
\begin{itemize}[leftmargin=1mm]
\setlength{\itemsep}{0pt}
\setlength{\parskip}{0.5pt}
\item sum 多组输入: 应只用了输入内容 (memset TL, 错下标 WA), 是否正确撤销
\item int/LL 溢出, INF/-1 大小,浮点数 eps 和 = 0
\item 类似 \texttt{pair <LL, int> x = pair <int, int>();} 不会报警告 
\item 离散化与二分: \texttt{lower\_bound upper\_bound +-1}
\item 自定义排序:排序方向,比较函数为小于,考虑坏点: 如叉积 (0, 0)
\item 样例是否为对称/回文的?考虑构造不对称的情况;考虑构造样例没有覆盖的情况
\item 复制过的代码对应位置正确修改
\item 变量重名,变量复用后忘记改回去
\item 分类讨论: if 嵌套, break 位置
\item 图: 边标号初始是否为 1, 单双向边, 反向边边权
\item 根号: 考虑根号两侧是否复杂度都对
\end{itemize}
\end{small}