\subsubsection*{ZJJ: SAM处理手法}
\noindent 1. 基本子串结构: CLB 搞的那玩意。\\	
\noindent 2. 正反串 SAM 的基本联系: 一个子串出现的位置将会在两个SAM中同时得到映照。\\	
\noindent 3. SAM 上转成数点问题。\\	
\noindent 4. 线段树合并维护 endpos 集合。\\	
\noindent 5. 树剖保证到根的链上只涉及 log 次修改和查询。(区间 border) \\	
\noindent 6. LCT 保证到根的链只修改均摊 log 个不同的颜色段。(区间本质不同子串数量)

\subsubsection*{ZJJ: 字符串常见错误}
\noindent 1. 字符串算法变式记得判匹配位置超出字符串的情况,例如多组数据下的双端插入回文串。
\noindent 2. 警惕四则运算中'a'和' a'的区别。